\section{Reflexion und Ausblick} % (fold)
\label{sec:reflexion}

Insgesamt ist das Projekt sehr weit so verlaufen, wie ich es mir vorgestellt hatte und das Ergebnis finde ich auch durchaus gelungen. Einige Erweiterungen, die ich in dem Spiel gerne haben würde, um das Spielerlebnis noch einzigartiger zu gestalten, konnten leider noch nicht umgesetzt werden, aber der Grundstein dafür ist gelegt. Die wichtigsten Features sind dabei die intelligenteren Kreaturenarten bzw. andere Wegsuchealgorithmen, wie sie in \ref{sub:strategy_pattern} beschrieben wurden. Des Weiteren steht noch die Implementierung eines Kampagnenmodus und eines Tutoriallevels an, um so einen gewissen Fortschritt durch das Spielen zu bekommen. Eine größere Herausforderung ist, ein Game-Balancing-System umzusetzen, da dafür eine Menge Daten und Datenverarbeitung nötig wären. Im Moment sind die Schwierigkeitsüberlegungen alle manuell fest einprogrammiert. 

Eine persönliche Weiterbildung gelang vor allem in der View-Programmierung, da es mein erstes JavaFX Projekt ist. Das Framework hat viele Dinge vereinfacht, mit denen ich mich bei Swing sonst lange herumschlagen musste, unter anderem Layouting mit fxml und Drehungen von Elementen und Animationen. Insbesondere die Bindings sind großartig und entlasten das MVC-Konzept (was man an den kleinen Controller-Klassen sieht). Auch die Kompatibilität zu Android ist ein großes Plus und macht es auch für Programmierprojekte im Schulunterricht nochmal interessanter.

Die konsequent testgetriebene Entwicklung benötigte während der Implementierung deutlich merklich mehr Zeit. Jedoch hatte ich dadurch während des ganzen Projektes auch bei größeren Änderungen nie Probleme Fehler zu finden. Außerdem gab die große Testabdeckung eine gewisse Sicherheit, sodass ich vor den größeren Umstrukturierungen auch nicht verunsichert war, ob hinterher noch alles funktioniert. Diese Sachen haben viel Zeit wieder eingespielt und ich denke, dass dadurch auch eine höhere Code-Qualität zustande gekommen ist.

Negativ ist zu bemerken, dass ich die Javadoc-Dokumentation sehr habe schleifen lassen. Die konsequente Kommentierung des Codes kann auf jeden Fall noch verbessert werden. Dafür wurde versucht möglichst sprechenden Code (Variablen-, Methoden- und Klassennamen) zu schreiben und lange Methoden oder sogar Klassen in kleinere zu unterteilen.

Insgesamt lässt sich aber festhalten, dass das Projekt durchaus gelungen umgesetzt wurde und ohne größere Komplikationen verlief.